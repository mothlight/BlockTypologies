%%%%%%%%%%%%%%%%%%%%%%%%%%%%%%%%%%%%%%%%%
% Plain Cover Letter
% LaTeX Template
% Version 1.0 (28/5/13)
%
% This template has been downloaded from:
% http://www.LaTeXTemplates.com
%
% Original author:
% Rensselaer Polytechnic Institute 
% http://www.rpi.edu/dept/arc/training/latex/resumes/
%
% License:
% CC BY-NC-SA 3.0 (http://creativecommons.org/licenses/by-nc-sa/3.0/)
%
%%%%%%%%%%%%%%%%%%%%%%%%%%%%%%%%%%%%%%%%%

%----------------------------------------------------------------------------------------
%	PACKAGES AND OTHER DOCUMENT CONFIGURATIONS
%----------------------------------------------------------------------------------------

\documentclass[11pt]{letter} % Default font size of the document, change to 10pt to fit more text

\usepackage{newcent} % Default font is the New Century Schoolbook PostScript font 
%\usepackage{helvet} % Uncomment this (while commenting the above line) to use the Helvetica font

% Margins
\topmargin=-1in % Moves the top of the document 1 inch above the default
\textheight=8.5in % Total height of the text on the page before text goes on to the next page, this can be increased in a longer letter
\oddsidemargin=-10pt % Position of the left margin, can be negative or positive if you want more or less room
\textwidth=6.5in % Total width of the text, increase this if the left margin was decreased and vice-versa

\let\raggedleft\raggedright % Pushes the date (at the top) to the left, comment this line to have the date on the right

\begin{document}

%----------------------------------------------------------------------------------------
%	ADDRESSEE SECTION
%----------------------------------------------------------------------------------------

\begin{letter}{Nature Editors
% \\
%Recruitment Officer \\
%The Corporation \\
%123 Pleasant Lane \\
%City, State 12345
} 

%----------------------------------------------------------------------------------------
%	YOUR NAME & ADDRESS SECTION
%----------------------------------------------------------------------------------------




\begin{center}
\large\bf Dr. Kerry Nice and Dr. Gideon Aschwanden \\ % Your name
%\vspace{20pt} \hrule height 1pt % If you would like a horizontal line separating the name from the address, uncomment the line to the left of this text
Transport, Health and Urban Design (THUD) research hub \\
Faculty of Architecture, Building and Planning  \\
Room 415, Level 4, Melbourne School of Design (Building 133) \\
The University of Melbourne, Victoria 3010 Australia \\
T: +61 (03)834 41756, M: +61 (0)422 883 121 \\
E: kerry.nice@unimelb.edu.au % Your address and phone number
\end{center} 
\vfill

\signature{Kerry Nice and Gideon Aschwanden} % Your name for the signature at the bottom

%----------------------------------------------------------------------------------------
%	LETTER CONTENT SECTION
%----------------------------------------------------------------------------------------

\opening{Dear Nature Editors,} 
 
%In addition, a cover letter needs to be written with the
%following:
%\begin{enumerate}
% \item A 100 word or less summary indicating on scientific grounds
%why the paper should be considered for a wide-ranging journal like
%\textsl{Nature} instead of a more narrowly focussed journal.
% \item A 100 word or less summary aimed at a non-scientific audience,
%written at the level of a national newspaper.  It may be used for
%\textsl{Nature}'s press release or other general publicity.
% \item The cover letter should state clearly what is included as the
%submission, including number of figures, supporting manuscripts
%and any Supplementary Information (specifying number of items and
%format).
% \item The cover letter should also state the number of
%words of text in the paper; the number of figures and parts of
%figures (for example, 4 figures, comprising 16 separate panels in
%total); a rough estimate of the desired final size of figures in
%terms of number of pages; and a full current postal address,
%telephone and fax numbers, and current e-mail address.
%\end{enumerate}

% Reason why widespread
The presented project allows investigation of the structure of human settlements and proposes an objective, evidence based method to analyse individual urban structures. This allows other fields that use the urban environment as an input parameter, such as public health, urban economics, and transportation planing, to examine these areas with data at a new level of granularity not previously available to them.

% Non-Scientific audience
We can answer, is your neighbourhood similar to Soho in London or Kibera in Neirobi. This new methodology compares neighbourhoods of all cities, with more than 300,000 inhabitants, around the world. This method can enable many fields to discover insights about urban areas at a fine grained neighbourhood level. 

% Number of Fig. | manuscripts | Supplementary Information

- 1.7 million Image Data points
- Github


\closing{Kerry Nice and Gideon Aschwanden,}


\encl{Extended abstract} % List your enclosed documents here, comment this out to get rid of the "encl:"

%----------------------------------------------------------------------------------------

\end{letter}

\end{document}